\documentclass[12pt,pdftex,oneside]{article}
% Mes ajouts pour les accents
\usepackage[utf8]{inputenc}%
\usepackage[T1]{fontenc}%
\usepackage[french]{babel}%

\include{mes_macros}
%\include{graphicx}

\renewcommand{\theenumii}{\roman{enumii}}%

\begin{document}

  \section {Classe Belligérant}

  Un belligérant capable de combattre à mains nues (et en pagne).

  \begin{itemize}
  \item Propriétés : 
    \begin{enumerate}
    \item nom : Le nom du belligérant.
          \begin{itemize}
          \item Type : str
          \item Accès : lecture seule
          \end{itemize}
    \item pts\_vies : Le nombre de points de vie du belligérant.
          \begin{itemize}
          \item Type : int
          \item Accès : accès complet
          \end{itemize}
    \item défense : Le facteur de défense du belligérant. Plus ce facteur est
      élevé, plus le belligérant pourra résister aux attaques.
          \begin{itemize}
          \item Type : int
          \item Accès : lecture seule
          \end{itemize}
    \item force : Le facteur de force du belligérant. Plus ce facteur est
      élevé, plus les attaques du belligérant seront efficaces.
          \begin{itemize}
          \item Type : int
          \item Accès : lecture seule
          \end{itemize}

    \end{enumerate}

  \item Constructeur : 

    \begin{enumerate}
    \item \_\_init\_\_ : Initialise un belligérant. Outre le nom, fournit en
paramètre, les valeurs de force, défense et de points de vie sont calculés de la
façon suivante :
        \begin{itemize}
          \item force : 1 dé12
          \item défense : 1 dé12
          \item pts\_vie : 2 dé12 + 20
        \end{itemize}
    \end{enumerate}

  \item Méthodes : 

    \begin{enumerate}
    \item nom : \_\_str\_\_ : Fournit une chaîne de caractère représentant le belligérant.
      \begin{itemize}
      \item Retour : Une chaîne représentant le belligérant de la forme «nom
        (F:\emph{force} D:\emph{défense} V:\emph{pts\_vie})
          \begin{itemize}
          \item Type de retour: str
          \end{itemize}
      \end{itemize}

    \item attaquer : Calcule le coefficient d'attaque d'un assaut  par la formule.
      \begin{itemize}
      \item Retour : Le coefficient d'attaque d'un assaut calculé.
        : \emph{force} + 1 dé12
          \begin{itemize}
          \item Type de retour: int
          \end{itemize}
      \end{itemize}

    \item parer : Calcule le coefficient de parade lors d'un assaut.
      \begin{itemize}
      \item Retour : Le coefficient de parade lors d'un assaut calculé par la
        formule : \emph{défense} + 2 dé6
          \begin{itemize}
          \item Type de retour: int
          \end{itemize}
      \end{itemize}

    \item subir\_dégâts : Après avoir reçu une attaque, cette méthode calcule les
      dégâts subis par le belligérant et les déduit de ses points de vie selon
      la formule : dégâts = impact.
      \begin{itemize}
      \item Paramètres : 
        \begin{enumerate}
        \item impact (int) : La force d'impact de l'attaque reçue.
        \end{enumerate}
      \end{itemize}

    \item est\_mort : détermine si un belligérant est mort.
      \begin{itemize}
      \item Retour : Vrai si et seulement si pts\_vie est 0 ou moins.
          \begin{itemize}
          \item Type de retour: bool
          \end{itemize}
      \end{itemize}

  \end{itemize}

  \end{itemize}

  \section {Classe Équipe}

  Regroupe les belligérants d'une même équipe (d'un même joueur).

  \begin{itemize}
  \item Propriétés : 
    \begin{enumerate}
    \item nom : nom de l'équipe
          \begin{itemize}
          \item Type : str
          \item Accès : Lecture seule
          \end{itemize}

    \end{enumerate}

  \item Constructeur : 

  \begin{enumerate}
  \item \_\_init\_\_ : Initialise une équipe avec un nom mais sans belligérants
    immédiatement.
    \begin{itemize}
    \item Paramètres : 
      \begin{enumerate}
      \item nom (str) : Le nom de la nouvelle équipe.
      \end{enumerate}
    \end{itemize}

  \end{enumerate}

  \item Méthodes : 

    \begin{enumerate}
    \item \_\_len\_\_ : Retourne le nombre d'éléments dans l'équipe.
      \begin{itemize}
      \item Retour : Le nombre de belligérants dans l'équipe
          \begin{itemize}
          \item Type de retour: int
          \end{itemize}
      \end{itemize}

    \item ajouter\_belligérant : Ajoute un belligérant à l'équipe
      \begin{itemize}
      \item Paramètres : 
        \begin{enumerate}
        \item un\_belligérant : le nouveau belligérant à ajouter à l'équipe.
          \begin{itemize}
          \item Type : Belligérant
          \end{itemize}
        \end{enumerate}
      \end{itemize}
    \item belligérant : Accesseur des belligérants
      \begin{itemize}
      \item Paramètres : 
        \begin{enumerate}
        \item indice : numéro du belligérant à retourner
          \begin{itemize}
          \item Type : int
          \item Valeur par défaut : None
          \end{itemize}
        \item nom : nom du belligérant à retourner
          \begin{itemize}
          \item Type : str
          \item Valeur par défaut : None
          \end{itemize}
        \end{enumerate}
      \item Retour : Si \emp{indice} n'est pas None, retourne le belligérant numéro
        \emph{indice} dans la liste, sinon, retourne le belligérant dont le nom
        correspond à \emph{nom}. Si \emph{indice} et \emph{nom} sont None, retourne None.
          \begin{itemize}
          \item Type de retour: Belligérant
          \end{itemize}
      \item Assertions : 
        \begin{enumerate}
            \item \emph{indice} est None ou représente un élément existant de la liste de
            Belligérants
            \begin{itemize}
              \item Message : «indice \{\emph{indice}\} invalide»
            \end{itemize}
            \item \emph{nom} est None ou représente le nom d'un élément existant de la liste de
            Belligérants
            \begin{itemize}
              \item Message : «nom '\{\emph{indice}\}' est invalide»
            \end{itemize}
        \end{enumerate}
      \end{itemize}
    \end{enumerate}

  \end{itemize}
  \end{itemize}

  \section {Classe Guerrier}

  Un belligérant de type Guerrier. Il peut porter une armure et manier une arme.

  \begin{itemize}
  \item Hérite de : Belligérant
  \item Propriétés : 
    \begin{enumerate}
    \item armure : L'armure que porte actuellement le guerrier
          \begin{itemize}
          \item Type : Armure
          \item Accès : complet
          \end{itemize}
    \item arme : L'arme qu'utilise actuellement le guerrier
          \begin{itemize}
          \item Type : Arme
          \item Accès : complet
          \end{itemize}

    \end{enumerate}

  \item Constructeur : 

  \begin{enumerate}
  \item \_\_init\_\_ : Initialise un Guerrier, d'abord sans arme ni armure. Ses
    facteurs de force et de défense obtiennent un boni d'un dé12 par rapport au
    Belligérant et ses points de vie de 2xDé20.
    \begin{itemize}
    \item Paramètres : 
      \begin{enumerate}
      \item un\_nom (str) : Le nom du nouveau Guerrier
      \end{enumerate}
    \end{itemize}

  \end{enumerate}

  \item Méthodes : 

    \begin{enumerate}
    \item calculer\_dégâts : Surdéfinition de la méthode
      Belligérant.calculer\_dégâts(). En plus du calcul des dégâts comme pour
      n'importe quel belligérant, les dégâts réels sont réduits de le moitié de
      la classe de l'armure portée par le guerrier.
      \begin{itemize}
      \item Paramètres : 
        \begin{enumerate}
        \item impact (int) : La force d'impact de l'attaque reçue.
        \end{enumerate}
      \end{itemize}
      \item Retour : Le nombre de points de vie qui doivent être retirés au
        Guerrier suite à l'attaque.
          \begin{itemize}
          \item Type de retour: int
          \end{itemize}

    \item attaquer : Surdéfinition de la méthode Belligérant.attaquer. Le
      coefficient d'attaque du belligérant est multiplié par la classe de l'arme
      qu'il utilise au moment de l'attaque.
      \begin{itemize}
      \item Retour : Le coefficient d'attaque d'un assaut.
          \begin{itemize}
          \item Type de retour: int
          \end{itemize}
      \end{itemize}

    \item subir\_dégâts : Surdéfinition de la méthode
      Belligérant.subir\_dégâts(). Soustrait au Guerrier le nombre de points de
      vie calculé par calculer\_dégâts. L'armure subit ensuite autant
      d'usure qu'un cinquième de l'impact.
      \begin{itemize}
      \item Paramètres : 
        \begin{enumerate}
        \item impact (int) : La force d'impact de l'attaque reçue.
        \end{enumerate}
      \end{itemize}
    \end{enumerate}

  \end{itemize}

  \section {Classe Mage}

  Un belligérant de type mage capable de jeter des sorts.

  \begin{itemize}
  \item Hérite de : Belligérant
  \item Propriétés : 
    \begin{enumerate}
    \item puissance : La puissance du mage. Pour lancer un sort, il ne peut
      jeter que des sorts dont la classe est inférieure ou égale à sa puissance.
          \begin{itemize}
          \item Type : int
          \item Accès : complet
          \end{itemize}
    \item mana : La quantité d'«énergie» magique que possède le Mage. Il ne peut
      lancer de sort qui demande plus de mana que la quantité qu'il possède.
          \begin{itemize}
          \item Type : int
          \item Accès : complet
          \end{itemize}

    \end{enumerate}

  \item Constructeur : 

  \begin{enumerate}
  \item \_\_init\_\_ : Initialise un Mage. Sa puissance est donnée par un dé6 et
    sa mana par 2xDé20. Initialement, il ne possède aucun sort.
    \begin{itemize}
    \item Paramètres : 
      \begin{enumerate}
      \item param (un\_nom) : Le nom du Mage.
      \end{enumerate}
    \end{itemize}

  \end{enumerate}

  \item Méthodes : 

    \begin{enumerate}
    \item jeter\_sort : Jete un sort vers une cible. Pour réussir à lancer son sort, la
      différence entre la puissance du mage et la classe du sort doit être plus
      grande que le lancé d'un dé6. Dans tous les cas, la mana du mage est diminuée d'autant qu'il est requis par le sort.
      \begin{itemize}
      \item Paramètres : 
        \begin{enumerate}
        \item sort (Sort) : Le sort qui doit être jeté par le Mage
        \item cible (Belligérant) : Le belligérant vers qui le sort est jeté.
        \end{enumerate}
      \item Assertions : 
        \begin{enumerate}
        \item La puissance du Mage doit être au moins aussi grande que le classe
          de \emph{sort}
          \begin{itemize}
          \item Message : «Puissance (\emph{puissance}) < Sort.classe
            (\emph{sorte.classe})        
        \item La man du Mage doit être au moins aussi grande que la mana requise
          par le \emph{sort}
          \begin{itemize}
          \item Message : «Mana (\emph{mana}) < Sort.mana\_requise (\emph{sorte.mana\_requise})
          \end{itemize}
        \end{enumerate}
      \end{itemize}

    \item parer : Calcule le coefficient de parade lors d'un assaut. Puisque le
      Mage est plus agile que le Belligérant moyen, son coefficient est 10\% par
      point de puissance de plus que celui calculé par Belligérant.parer.
      \begin{itemize}
      \item Retour : Le coefficient de parade lors d'un assaut.
          \begin{itemize}
          \item Type de retour: int
          \end{itemize}
      \end{itemize}
    \end{enumerate}

    \item ajouter\_sort : Ajoute un sort à la liste des sorts connus par le Mage.
      \begin{itemize}
      \item Paramètres : 
        \begin{enumerate}
        \item un\_sort (sort) : Le nouveau Sort à ajouter au Mage.
        \end{enumerate}
      \item Assertions : 
        \begin{enumerate}
        \item Le mage ne doit pas déjà posséder ce sort
          \begin{itemize}
          \item Message : «Le Sort (\emph{un\_sort}) existe déjà»
          \end{itemize}
        \end{enumerate}
      \end{itemize}

  \end{itemize}

  \section {Classe Sort}

  Classe abstraite représentant un sort qui peut être jeté par un Mage.

  \begin{itemize}
  \item Propriétés : 
    \begin{enumerate}
    \item mana\_requise : quantité de mana requise pour jeter ce sort
          \begin{itemize}
          \item Type : int
          \item Accès : lecture seule
          \end{itemize}
    \item classe : la classe du sort, représente la difficulté à jeter ce sort.
          \begin{itemize}
          \item Type : int
          \item Accès : lecture seule
          \end{itemize}

    \end{enumerate}

  \item Méthodes : 

    \begin{enumerate}
    \item activer : Méthode abstraite. Active le sort qui agit alors sur sa cible.
      \begin{itemize}
      \item Paramètres : 
        \begin{enumerate}
        \item cible (Belligérant) : Le belligérant ciblé par le sort.
        \end{enumerate}
      \end{itemize}
    \end{enumerate}

  \end{itemize}

  \section {Classe SortDartDeFeu}

  Sort qui lance un dart de feu vers un belligérant.

  \begin{itemize}
  \item Hérite de : Sort


  \item Constructeur : 

  \begin{enumerate}
  \item \_\_init\_\_ : Initialise le Sort avec sa classe et sa mana.
    \begin{itemize}
    \item Paramètres : 
      \begin{enumerate}
      \item une\_classe (int) : La classe du sort (1)
      \item une\_mana\_requise (int) : La quantité de mana requise par le sort (2)
      \end{enumerate}
    \end{itemize}

  \end{enumerate}

  \item Méthodes : 

    \begin{enumerate}
    \item activer : Active le sort qui agit alors sur sa cible. Il porte une
      attaque de puissance de 2xDé12. La cible peut parer le sort comme
      pour une attaque physique.
      \begin{itemize}
      \item Paramètres : 
        \begin{enumerate}
        \item cible (Belligérant) : Le belligérant ciblé par le sort.
        \end{enumerate}
      \end{itemize}
    \end{enumerate}

  \end{itemize}

  \section {Classe SortGuérison}

  Sort qui permet d'augmenter le nombre de points de vie d'un belligérant.

  \begin{itemize}
  \item Hérite de : Sort


  \item Constructeur : 

  \begin{enumerate}
  \item \_\_init\_\_ : Initialise le Sort avec sa classe et sa mana.
    \begin{itemize}
    \item Paramètres : 
      \begin{enumerate}
      \item une\_classe (int) : La classe du sort (1)
      \item une\_mana\_requise (int) : La quantité de mana requise par le sort (4)
      \end{enumerate}
    \end{itemize}

  \end{enumerate}

  \item Méthodes : 

    \begin{enumerate}
    \item activer : Active le sort qui agit alors sur sa cible. Il porte redonne
      1 dé20 de points de vie à sa cible.
      \begin{itemize}
      \item Paramètres : 
        \begin{enumerate}
        \item cible (Belligérant) : Le belligérant ciblé par le sort.
        \end{enumerate}
      \end{itemize}
    \end{enumerate}

  \end{itemize}

  \section {Classe SortProtection}

  Sort qui augmente la protection d'un belligérant.

  \begin{itemize}
  \item Hérite de : Sort


  \item Constructeur : 

  \begin{enumerate}
  \item \_\_init\_\_ : Initialise le Sort avec sa classe et sa mana.
    \begin{itemize}
    \item Paramètres : 
      \begin{enumerate}
      \item une\_classe (int) : La classe du sort (2)
      \item une\_mana\_requise (int) : La quantité de mana requise par le sort (8)
      \end{enumerate}
    \end{itemize}

  \end{enumerate}

  \item Méthodes : 

    \begin{enumerate}
    \item activer : Active le sort qui agit alors sur sa cible. Il augmente la
      défense du belligérant de 50\%.
      \begin{itemize}
      \item Paramètres : 
        \begin{enumerate}
        \item cible (Belligérant) : Le belligérant ciblé par le sort.
        \end{enumerate}
      \end{itemize}
    \end{enumerate}

  \end{itemize}

  \section {Classe SortRésurrection}

  Description

  \begin{itemize}
  \item Hérite de : Sort


  \item Constructeur : 

  \begin{enumerate}
  \item \_\_init\_\_ : Initialise le Sort avec sa classe et sa mana.
    \begin{itemize}
    \item Paramètres : 
      \begin{enumerate}
      \item une\_classe (int) : La classe du sort (4)
      \item une\_mana\_requise (int) : La quantité de mana requise par le sort (15)
      \end{enumerate}
    \end{itemize}

  \end{enumerate}

  \item Méthodes : 

    \begin{enumerate}
    \item activer : Active le sort qui agit alors sur sa cible. Il redonne à sa
      cible 2xDé20 points de vie.
      \begin{itemize}
      \item Paramètres : 
        \begin{enumerate}
        \item cible (Belligérant) : Le belligérant ciblé par le sort.
        \end{enumerate}
      \end{itemize}
    \end{enumerate}

  \end{itemize}

  \section {Classe Arme}

  Classe abstraite représentant une arme générique.

  \begin{itemize}
  \item Propriétés : 
    \begin{enumerate}
    \item classe : La classe de l'arme. Plus la classe est élevée, plus elle est destructrice.
          \begin{itemize}
          \item Type : int
          \item Accès : lecture seule
          \end{itemize}
    \item bonus : Le bonus d'attaque accordé par l'arme. Cette propriété est
      calculée par les sous-classes.
          \begin{itemize}
          \item Type : int
          \item Accès : lecture seule
          \end{itemize}

    \end{enumerate}

  \item Constructeur : 

  \begin{enumerate}
  \item \_\_init\_\_ : Initialise un Arme avec sa classe.
    \begin{itemize}
    \item Paramètres : 
      \begin{enumerate}
      \item une\_classe (int) : La classe de l'arme.
      \end{enumerate}
      \item Assertions : 
        \begin{enumerate}
        \item La classe ne peut être négative
          \begin{itemize}
          \item Message : «La classe (\emph{une\_classe}) est invalide»
          \end{itemize}
        \end{enumerate}
    \end{itemize}

  \end{enumerate}

  \item Méthodes : 

    \begin{enumerate}
    \item \_\_str\_\_ : Méthode abstraite qui retourne une représentation en chaîne de caractère de l'arme
      sous la forme «Type d'arme (classe)».
      \begin{itemize}
      \item Retour :  une représentation en chaîne de caractère de l'arme
      sous la forme «Type d'arme (classe)».
          \begin{itemize}
          \item Type de retour: str
          \end{itemize}
      \end{itemize}
    \item bonus : Méthode abstraite qui calcule le bonus accordé par cette arme
      pour une attaque.
      \begin{itemize}
      \item Retour : Le bonus accordé par cette arme.
          \begin{itemize}
          \item Type de retour: int
          \end{itemize}
      \end{itemize}
    \end{enumerate}

  \end{itemize}

  \section {Classe Armure}

  Classe abstraite représentant une armure générique.

  \begin{itemize}
  \item Propriétés : 
    \begin{enumerate}
    \item classe : La classe de l'armure. Plus la classe est élevée, plus elle
      protège celui qui la porte.
          \begin{itemize}
          \item Type : int
          \item Accès : lecture seule
          \end{itemize}
    \item bonus : Le bonus de défense accordé par l'arme. Cette propriété est
      calculée par les sous-classes.
          \begin{itemize}
          \item Type : int
          \item Accès : lecture seule
          \end{itemize}
    \item usure : Le pourcentage d'usure de l'armure. Initialement à 0, elle
      augent progressivement jusqu'à 100; l'armure est alors rendue inutile.
          \begin{itemize}
          \item Type : int
          \item Accès : complet
          \end{itemize}

    \end{enumerate}

  \item Constructeur : 

  \begin{enumerate}
  \item \_\_init\_\_ : Initialise un Armure avec sa classe.
    \begin{itemize}
    \item Paramètres : 
      \begin{enumerate}
      \item une\_classe (int) : La classe de l'armure.
      \end{enumerate}
      \item Assertions : 
        \begin{enumerate}
        \item La classe ne peut être négative
          \begin{itemize}
          \item Message : «La classe (\emph{une\_classe}) est invalide»
          \end{itemize}
        \end{enumerate}
    \end{itemize}

  \end{enumerate}

  \item Méthodes : 

    \begin{enumerate}
    \item \_\_str\_\_ : Méthode abstraite qui retourne une représentation en chaîne de caractère de l'armure.
      sous la forme «Type d'armure (classe)».
      \begin{itemize}
      \item Retour :  une représentation en chaîne de caractère de l'armure
      sous la forme «Type d'armure (classe)».
          \begin{itemize}
          \item Type de retour: str
          \end{itemize}
      \end{itemize}
    \item bonus : Méthode abstraite qui calcule le bonus accordé par cette armure
      pour une attaque.
      \begin{itemize}
      \item Retour : Le bonus accordé par cette armure.
          \begin{itemize}
          \item Type de retour: int
          \end{itemize}
      \end{itemize}
    \end{enumerate}

  \end{itemize}

  \section {Classe Dé}

  Description

  \begin{itemize}

  \item Méthodes : 

    \begin{enumerate}
    \item lancer : Méthode de classe qui simule un lancer de dé.
      \begin{itemize}
      \item Paramètres : 
        \begin{enumerate}
        \item faces (int) : le nombre de faces du dé à lancer.
          \begin{itemize}
          \item Valeur par défaut : 6
          \end{itemize}
      \end{enumerate}
      \item Retour : Un nombre au hasard entre 1 et \emph{faces} inclusivement.
          \begin{itemize}
          \item Type de retour: int
          \end{itemize}
      \item Assertions : 
        \begin{enumerate}
        \item \emph{faces} doit être > 1
          \begin{itemize}
          \item Message : «Le nombre de faces doit être > 1»
          \end{itemize}
        \end{enumerate}
      \end{itemize}
    \end{enumerate}

  \end{itemize}

\end{document}

